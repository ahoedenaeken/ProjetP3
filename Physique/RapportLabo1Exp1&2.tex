\documentclass[11pt,a4paper]{report}
\usepackage[utf8]{inputenc}
\usepackage[francais]{babel}
\usepackage[T1]{fontenc}
\usepackage{amsmath}
\usepackage{amsfonts}
\usepackage{amssymb}
\usepackage{graphicx}
\author{Groupe 1246}
\begin{document}
\section{Rapport pts 1 et 2}
Nous avons commencé par mesurer les angles de réflexion et de réfraction pris par le faisceau dans plusieurs situations.Premièrement nous avons commencé avec le cas où les rayons était réfléchis sans passer dans le milieu de la lucite Nos résultats sont repris dans le tableau suivant:

\begin{tabular}{|c|c|c|}
\hline 
Angle réfléchis & Angle réfracté & n_2 \\ 
\hline 
10\degre & 7\degre & 1.42 \\ 
\hline 
20\degre & 13\degre & 1.52 \\ 
\hline 
30\degre & 21\degre & 1.39 \\ 
\hline 
40\degre & 26\degre & 1.46 \\ 
\hline 
50\degre & 30\degre & 1.53 \\ 
\hline 
60\degre & 35\degre & 1.51 \\ 
\hline 
70\degre & 37.5\degre & 1.54 \\ 
\hline 
80\degre & 40\degre & 1.53 \\ 
\hline 
\end{tabular} 

La loi de Snell Descartes (pour rappel: $n_1 \cdot sin_{\Theta _{1}} =n_2 \cdot sin_{\Theta _2}$)est bien vérifiée et les résultats de calcul de l'indice de réfraction ($n_{2}$) de la lucite sont repris dans le tableau. 

Nous pensons que le manque de précision dans le calcul du $n_{2}$ pour des petits angles vient du fait que nos mesures de l'angle réfracté n'étant pas parfaitement précise, l'erreur se fait plus sentir lors des calculs pour les petits angles. Autrement:

soit $\Theta _2$ l'angle de réfraction obtenu

soit $i$ l'imprecision de mesure

soit $\Theta _t$ l'angle de réfraction théorique

on a : $\Theta _2 =\Theta _t + i $

pour des petits angles on a : $sin \alpha \approx \alpha$

L'erreur $i$ a donc une plus grande influence pour des calculs sur des petits angles que sur des angles plus grand, avec lesquels elle devient presque insignifiante.


Nous avons ensuite effectué les même mesures d'angle lorsque la face arrondi du polymère était placée vers le laser. Les résultats sont les suivant:



Nous avons déterminé que l'angle critique était de 43\degre .

on sait que:

$sin(\Theta _crit) = \dfrac{n_2}{n_1} = \dfrac{n_{air}}{n_{luc}}$ 

de la on trouve que $n_{luc} = 1.46$

on trouve alors que l'erreur pour un n_{théo} valant 1.49 vaut 2.01 %.
 
Utiliser un prisme hémi-cylindrique permet de garder le rayon complet, sans réfraction, jusqu'à son point de réflexion. On peut donc faire les calcul comme si le rayon évoluait dans l'air. Cela évite aussi une modifiaction de l'angle incident.
\end{document}