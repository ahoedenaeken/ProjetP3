
\section{Bilan d'énergie}

Cette section traite le bilan énergétique du procédé étudié.
Sur base du flow-sheet simplifié disponible ci-après, nous avons déterminé que seulement trois réactions
devaient être analysées pour réaliser un tel bilan: celle se produisant dans le bloc du four et celles dans 
le réacteur du reformage primaire. Grâce à cette analyse, nous obtiendrons une expression pour la quantité de \ce{CH4} nécessaire pour la production d'une certaine quantité d'ammoniac, à une certaine température.

\subsection{Le bloc "Réacteur de reformage primaire"}

Dans ce bloc, nous devons traiter les deux réactions à l'équilibre suivantes, prenant place simultanément, à une certaine température $T$:

\begin{equation}\label{eq:Reaction1}
\ce{CH_{4(g)} + H_{2}O_{(g)} <=> 3H_{2(g)} + CO_{(g)}}
\end{equation}
\begin{equation}\label{eq:Reaction2} {\ce{CO_{(g)} + H_{2}O_{(g)} <=>  H_{2(g)} + CO_{2(g)}}}
\end{equation}


\bigbreak
Nous allons développer le raisonnement que nous avons effectué afin d'obtenir l'expression de la différence d'enthalpie dans le réacteur et, dans un premier temps, nous ne nous intéresserons qu'à la première des deux réactions. Fondamentalement, nous savons que:
$$\Delta H\degree_{Reaction1}(T)=3\Delta H\degree_{H_{2(g)}}(T) + \Delta H\degree_{CO_{(g)}}(T) -
\lbrack\Delta H\degree_{CH_{4(g)}}(T) + \Delta H\degree_{H_{2}O_{(g)}}(T)\rbrack$$

Pour rappel, l'enthalpie standard d'une espèce chimique à une certaine température $T$ s'écrit: \begin{equation}\label{eq:deltaH}
\Delta H\degree(T)=\Delta H\degree(T_{ref})  + \displaystyle \int_{T_{ref}}^{T} C_{p}(T) \mathrm{d}T
\end{equation}


Les enthalpies à une température de référence (typiquement, $T_{ref}=\unit{25}{\celsius}=\unit{298}{\kelvin}$) sont disponibles dans des tables de mesures expérimentales. Il ne nous reste donc plus qu'à trouver l'expression de $C_p$, la capacité thermique à pression constante, en fonction de $T$.
Après quelques recherches\cite{NIST}, nous avons trouvé la formule suivante: \begin{equation}\label{eqref:capacite}
C_p=A+\dfrac{BT}{1000}+C\left(\dfrac{T}{1000}\right)^2+D\left(\dfrac{T}{1000}\right)^3+E\left(\dfrac{1000}{T}\right)^2
\end{equation} où $A,B,C,D,E$ sont des constantes propres à chaque composant (cfr. Annexe).
En remplaçant, nous pouvons facilement obtenir une expression pour $\Delta H\degree_{Reaction1}$, que nous ne retranscrirons pas
ici, mais qui est disponible dans le code \textsc{MATLAB} en Annexe. 
Les calculs pour la deuxième réaction à l'équilibre étant similaires, nous pouvons
calculer $\Delta H\degree_{Reaction2}(T)$ sans difficulté. A fin de calculer la différence d'enthalpie dans le réacteur
il nous faut connaître les degrés d'avancement des réactions calculées dans le bilan de matière. Ainsi, nous trouvons la relation suivante, où $m$ et $n$ sont respecitvement les degrés d'avancement de la réaction (\ref{eq:Reaction1}) et de la réaction (\ref{eq:Reaction2}):
\begin{equation}\label{eq:deltaHreacteur}\Delta H\degree_{Reacteur}(T)=m\cdot\Delta H\degree_{Reaction1}(T)+n\cdot\Delta H\degree_{Reaction2}(T)}\end{equation}

\subsection{Le bloc "Four"}
La réaction ayant lieu dans le four n'est autre que la combustion du méthane:

\begin{equation}\label{combust.methane}\ce{CH_{4(g)} + 2O_{2(g)} <=> 2H_2O_{(g)} + CO_{2(g)}}\end{equation}

De plus, nous savons que:

$$\Delta H\degree_{Four} = 2\cdot \Delta H\degree_{H_{2}O_{(g)}} + \Delta H\degree_{CO_{2(g)}}
- \left(\Delta H\degree_{CH_{4(g)}} + 2\cdot \Delta H\degree_{O_{2(g)}}\right)$$

Nous considèrerons que la réaction du four se passe également à la température $T$. 
Nous devons donc à nouveau trouver la valeur du $\Delta H\degree(T)$ de chaque composé grâce à la formule (\ref{eq:deltaH}),
puis injecter ces valeurs dans l'équation du $\Delta H\degree_{Four}$ -- ces calculs, vu leur longueur, ne sont pas repris dans
ce document. Nous supposons que la réaction à la température $T$ sera exothermique car le four sert à fournir de l'énergie à la réaction
endothermique du réacteur primaire de reformage. Nous posons: $\Delta H_{Reacteur} = a$ (trouvé à l'aide de la relation (\ref{eq:deltaHreacteur})), $\Delta H_{Four} = b$, 
et $DB$ la quantité de \ce{CH_4} nécessaire dans le four.
On trouve alors:

$DB = \dfrac{a}{b}\unit{}{\mole\per jour}$

