\section{Analyse paramétrique}

Après avoir conçu l'outil de gestion du plant, il est intéressant d'analyser les résultats obtenus en faisant varier les 
paramètres dans une plage réaliste. Le tableau ci-dessous reprend cette analyse. Nous avons considéré une variation de 
température de \unit{700}{\kelvin} à \unit{1300}{\kelvin}, étant donné que les valeurs de référence utilisées dans le 
calcul de $K(T)$ sont pour la plupart valables jusqu'à \unit{1300}{\kelvin}. En ce qui concerne la quantité d'ammoniac 
produite, nous avons testé de \unit{1000}{\ton} à \unit{2000}{\ton}, avec un pas de \unit{100}{\ton}. C'est la capacité 
typique d'un plant d'ammoniac par jour\footnote{voir Tache_1-suite_S2.pdf, iCampus}.
