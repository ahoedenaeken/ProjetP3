\section{Analyse paramétrique}

\paragraph{} Après avoir conçu l'outil de gestion du plant, il est intéressant d'analyser les résultats obtenus en 
faisant varier les paramètres dans une plage réaliste. Les tableaux ci-dessous reprennent cette analyse. Nous avons 
considéré une variation de température de \unit{800}{\kelvin} à \unit{1300}{\kelvin}. Ce choix a été déterminé par 
les valeurs de référence utilisées dans le calcul de $K(T)$, qui sont pour la plupart valables jusqu'à 
\unit{1300}{\kelvin}. En ce qui concerne la quantité d'ammoniac produite, nous avons testé de \unit{1000}{\ton} à 
\unit{2000}{\ton}, étant donné que c'est la capacité typique d'un plant d'ammoniac.


\paragraph{} Le premier tableau fait varier la température, en considérant la quantité d'ammoniac \ce{NH_3} produite
par jour constante et valant \unit{1500}{\meter^3\per jour}.
Le deuxième tableau fait varier la quantité d'ammoniac \ce{NH_3}, en considérant la température $T$ constante et 
valant \unit{1000}{\kelvin}.

\begin{table}[h]
\centering
\begin{tabular}{|c||c|c|c|}
\hline
Température & Débit de méthane \ce{CH_4} & Débit de l'eau \ce{H_{2}O} & Débit d'air \\ 
 & [\unit{\meter^3\per jour}] & [\unit{\meter^3\per jour}] & [\unit{\meter^3\per jour}] \\
\hline
800 & 86373.59 & 1044242.84 & 1.2518e+05 \\
\hline
900 & 97170.29 & 491179.16 & 1.4083e+05 \\
\hline
1000 & 107966.99 & 234055.84 & 1.5647e+05 \\
\hline
1100 & 118763.69 & 105168.26 & 1.7212e+05 \\
\hline
1200 & 129560.39 & 62398.32 & 1.8777e+05 \\
\hline
1300 & 140357.09 & 57193.98 & 2.0342e+05 \\
\hline
\end{tabular}
\caption{La température $T$ varie, la quantité d'ammoniac \ce{NH_3} produite par jour est constante}
\label{tab:tvarie}
\end{table}


\begin{table}[h]
\centering
\begin{tabular}{|c||c|c|c|}
\hline
Production journalière de \ce{NH_3} & Débit de méthane \ce{CH_4} & Débit de l'eau \ce{H_{2}O} & Débit d'air  \\ &
[\unit{\meter^3\per jour}] & [\unit{\meter^3\per jour}] & [\unit{\meter^3\per jour}] \\
\hline
\hline
1000 & 71978.00 & 156037.22 & 1.0432e+05 \\
\hline
1100 & 79175.80 & 171640.95 & 1.1475e+05 \\
\hline
1200 & 86373.59 & 187244.67 & 1.2518e+05 \\
\hline
1300 & 93571.39 & 202848.39 & 1.3561e+05 \\
\hline
1400 & 100769.19 & 218452.11 & 1.4604e+05 \\
\hline
1500 & 107966.99 & 234055.84 & 1.5647e+05 \\
\hline
1600 & 115164.79 & 249659.56 & 1.6691e+05 \\
\hline
1700 & 122362.59 & 265263.28 & 1.7734e+05 \\
\hline
1800 & 129560.39 & 280867.00 & 1.8777e+05 \\
\hline
1900 & 136758.19 & 296470.73 & 1.9820e+05 \\
\hline
2000 & 143955.99 & 312074.45 & 2.0863e+05 \\
\hline
\end{tabular}
\caption{La quantité d'ammoniac \ce{NH_3} produite par jour varie, la température $T$ est constante}
\label{tab:nh3varie}
\end{table}

Nous devions surtout surveiller la quantité d'eau , étant donné qu'elle devait être en excès dans les réactions à
l'équilibre, pour qu'il reste assez de vapeur pour permettre une conversion complète de  \ce{CO} en \{CO2} dans le 
réacteur Water-Gas-Shift.
Nous avons donc déterminé la température critique à partir de laquelle la quantité d'eau sortante devenait nulle. 
Nous sommes arrivés à la conclusion qu'il faut limiter la température de sortie du reformeur primaire à 
\unit{1049}{\kelvin}, si nous voulons une capacité d'ammoniac de \unit{1500}{\ton\per jour}.
