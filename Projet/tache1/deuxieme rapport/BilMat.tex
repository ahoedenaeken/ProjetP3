
%laissez sans \usepackage et \begin svp
\section{Bilan de matière}

\subsection{Introduction et notations}

Cette section met en avant l'évolution des réactifs et produits en terme de débit. 
Nous pouvons ainsi gérer les quantités de matière nécessaires à la réaction par unité de temps, et
moduler la production d'ammoniac. Nous avons procédé de manière systématique, en partant de la première
équation jusqu'à la dernière, et en posant quelques 
variables intermédiaires:

\begin{itemize}
	\item Soit $F_{mol, \ce{CH4}}$ le débit molaire de \ce{CH4}.
	\item Soit $F_{mol, \ce{H2O}}$ le débit molaire d'\ce{H2O}.
	\item Soit $\xi_1$ le degré d'avancement de la réaction à l'équilibre (1).
	\item Soit $\xi_2$ le degré d'avancement de la réaction à l'équilibre (2).
	\item Soit $F_{mol, \text{air}}$ le débit molaire d'air.
\end{itemize}

\subsection{Analyse et commentaires}
Dans la phase de reformage primaire, les réactions sont à l'équilibre; ce qui signifie qu'il faudra prendre en 
compte un degré d'avancement lié à la température par la constante d'équilibre. Pour le moment, nous ne devons pas 
nous en préoccuper, et nous poserons seulement les variables $\xi_1$ et $\xi_2$ identifiées ci-dessus. Pour raccourcir
l'écriture, nous abrégons le nombre de moles initial et final repectivement par $n_i$ et $n_f$.

\begin{figure}[h]
\begin{center}
\begin{tabular}{|c|c|c|c|c|}
\hline
&
\multicolumn{1}{c!{\makebox[0pt]{+}}}{
\ce{CH4}}
&
\multicolumn{1}{c!{\makebox[0pt]{$\rightleftharpoons$}}}{\ce{H2O}}
&
\multicolumn{1}{c!{\makebox[0pt]{+}}}{\ce{3H2}}
& \ce{CO}
\\
\hline
$n_i$ & $F_{mol, \ce{CH4}}$ & $F_{mol, \ce{H2O}}$ & $0$ & $0$ \\
\hline
$n_f$ & $F_{mol, \ce{CH4}}-\xi_1$ & $F_{mol, \ce{H2O}}-\xi_1-\xi_2$ & $3\xi_1+\xi_2$ & $\xi_1-\xi_2$ \\\hline
\end{tabular}
\end{center}
\caption{Tableau d'avancement de la réaction 1 du reformage primaire.}
\end{figure}
\begin{figure}[h]
\begin{center}
\begin{tabular}{|c|c|c|c|c|}
\hline
&
\multicolumn{1}{c!{\makebox[0pt]{+}}}{
\ce{CO}}
&
\multicolumn{1}{c!{\makebox[0pt]{$\rightleftharpoons$}}}{\ce{H2O}}
&
\multicolumn{1}{c!{\makebox[0pt]{+}}}{\ce{H2}}
& \ce{CO2}
\\
\hline
$n_i$ & $\xi_1$ & $F_{mol, \ce{H2O}}-\xi_1$ & $3\xi_1$ & $0$\\
\hline
$n_f$ & $\xi_1-\xi_2$ & $F_{mol, \ce{H2O}}-\xi_1-\xi_2$ & $3\xi_1+\xi_2$ & $\xi_2$ \\\hline
\end{tabular}
\end{center}
\caption{Tableau d'avancement de la réaction 2 du reformage primaire.}
\end{figure}

Les réactions prenant place dans la suite du procédé sont toutes considérées comme complètes dans cette Tâche \numprint{1}. Dans le reformeur secondaire, le méthane et l'oxygène
sont alimentés de façon stœchiométrique, ce qui signifie qu'il ne reste plus aucun de ces deux réactifs après réaction.
Tout a été converti. Cette condition nous permet d'aboutir à une première équation :
$$F_{mol, \ce{CH4}} - \xi_1 - 2 \cdot 0.21 \cdot F_{mol, \text{air}} = 0$$

\begin{figure}[h]
\begin{center}
\begin{tabular}{|c|c|c|c|c|}
\hline
&
\multicolumn{1}{c!{\makebox[0pt]{+}}}{
\ce{2CH4}}
&
\multicolumn{1}{c!{\makebox[0pt]{$\rightarrow$}}}{\ce{O2}}
&
\multicolumn{1}{c!{\makebox[0pt]{+}}}{\ce{2CO}}
& \ce{4H2}
\\
\hline
$n_i$ & $F_{mol, \ce{CH4}}-\xi_1$ & $0.21 \cdot F_{mol, \text{air}}$ & $\xi_1-\xi_2$ & $3\xi_1+\xi_2$\\
\hline
$n_f$ & $0$ & $0$ & $F_{mol, \ce{CH4}}-\xi_2$ & $\xi_1+\xi_2+2F_{mol, \ce{CH4}}$ \\\hline
\end{tabular}
\end{center}
\caption{Tableau d'avancement de la réaction du reformage secondaire}
\end{figure}

Dans le réacteur Water-Gas-Shift, tout le \ce{CO} aura réagi, mais il faudra séparer
l'eau restante, et absorber le \ce{CO2}. Nous obtenons l'expression du débit molaire d'eau sortant grâce au tableau .

\begin{figure}[h]
\begin{center}
\begin{tabular}{|c|c|c|c|c|}
\hline
&
\multicolumn{1}{c!{\makebox[0pt]{+}}}{
\ce{CO}}
&
\multicolumn{1}{c!{\makebox[0pt]{$\rightarrow$}}}{\ce{H2O}}
&
\multicolumn{1}{c!{\makebox[0pt]{+}}}{\ce{H2}}
& \ce{CO2}
\\
\hline
$n_i$ & $F_{mol, \ce{CH4}}-\xi_2$ & $F_{mol, \ce{H2O}}-\xi_1-\xi_2$ & $\xi_1+\xi_2+2F_{mol, \ce{CH4}}$ & $\xi_2$\\
\hline
$n_f$ & $0$ & $F_{mol, \ce{H2O}}-\xi_1-F_{mol, \ce{CH4}}$ & $\xi_1+3F_{mol, \ce{CH4}}$ & $F_{mol, \ce{CH4}}$ \\\hline
\end{tabular}
\end{center}
\caption{Tableau d'avancement de la réaction dans les réacteurs Water-Gas-Shift.}
\end{figure}

Enfin, la dernière réaction permet la synthèse de \ce{NH3}. Nous imposons qu'il
ne reste ni de \ce{N2}, ni de \ce{H2}, étant donné que la réaction globale peut être considérée
comme complète, grâce au "recyclage".
Nous obtenons ici une seconde équation:
$$ 2 \cdot 0.78 \cdot F_{mol, \text{air}} = \frac{2}{3}\cdot (\xi_1 + 3F_{mol, \ce{CH4}}) $$

\begin{figure}[h]
\begin{center}
\begin{tabular}{|c|c|c|c|}
\hline
&
\multicolumn{1}{c!{\makebox[0pt]{+}}}{
\ce{N2}}
&
\multicolumn{1}{c!{\makebox[0pt]{$\rightarrow$}}}{\ce{3H2}}
&
\ce{2NH3}
\\
\hline
$n_i$ & $0.78 \cdot F_{mol, \text{air}}$ & $\xi_1+3F_{mol, \ce{CH4}}$ & $0$ \\
\hline
$n_f$ & $0$ & $0$ & $2 \cdot 0.78 \cdot F_{mol, \text{air}}$ \\\hline
\end{tabular}
\end{center}
\caption{Tableau d'avancement de la synthèse de l'ammoniac}
\end{figure}

Au terme de ce procédé, nous aurons donc un débit molaire de prouction valant $1.56 \cdot F_{mol, \text{air}}$ de \ce{NH3}. Cette
quantité est liée au débit molaire de \ce{CH4} introduites dans le réacteur (i.e. $F_{mol, \ce{CH4}}$) par les deux équations obtenues plus haut:
\[
\left \{
\begin{array}
& F_{mol, \ce{CH4}} - \xi_1 - 2\cdot0.21 \cdot F_{mol, \text{air}} = 0
& 2\cdot0.78 \cdot F_{mol, \text{air}} = \frac{2}{3}\cdot (\xi_1 + 3F_{mol, \ce{CH4}}) 
\end{array}
\right.
\]
En simplifiant, on obtient: $4 \cdot F_{mol, \ce{CH4}} = 2.76 \cdot F_{mol, \text{air}}$. Nous avons maintenant la quantité de \ce{CH4} introduite
dans le réacteur par jour en fonction de la quantité d'ammoniac produite. Notons que la température n'a pas d'effet
sur cette quantité. Mais nous n'avons pas encore d'information à propos de la quantité d'eau nécessaire.


Afin d'obtenir une relation avec le nombre de moles d'\ce{H2O} introduites dans le réacteur (i.e. $F_{mol, \ce{H2O}}$), nous
devons utiliser les constantes d'équilibre des deux premières réactions. Pour une certaine
température $T$ [\unit{}{\kelvin}], nous avons l'expression suivante pour $K_1 (T)$ et $K_2 (T)$, respectivement les
constantes d'équilibre de la première et de la deuxième équation, avec $p_{tot}$ la pression totale dans le système et $(p\degree)^2$ la
pression standard de \unit{1}{bar}:

\[
\left \{
\begin{array}
& K_1 (T) = \left( \dfrac{ (3\xi_1 + \xi_2)^3 \cdot (\xi_1 - \xi_2) \cdot p_{tot}^3 }{ (F_{mol, \ce{CH4}} + 2\xi_1 + F_{mol, \ce{H2O}}) ^2 \cdot (F_{mol, \ce{H2O}} - \xi_1 - \xi_2) \cdot (F_{mol, \ce{CH4}}-\xi_1) \cdot p\degree^2}\right)
\\
& K_2 (T) = \left( \dfrac{(3\xi_1 + \xi_2)\cdot n}{(\xi_1-\xi_2)\cdot(F_{mol, \ce{H2O}}-\xi_1-\xi_2)} \right)
\end{array}
\right.
\]

Nous pouvons calculer la valeur théorique de ces constantes d'équilibre en fonction de la température
avec \textsc{Matlab} (voir code en Annexes), en posant la pression du reformeur primaire
à \unit{30}{\bbar}\footnote{Valeur conseillée sur iCampus, consulté le mercredi 3 octobre 2014}.
Il nous reste donc \numprint{4} équations à \numprint{4} inconnues et \numprint{2} paramètres: la
température $T$ et $F_{mol, \text{air}}$ le nombre de moles d'air introduites dans le réacteur. À partir de ces équations, nous avons
créé un outil de gestion permettant de réaliser le calcul d'approvisionnement nécessaire en matière première, en
fonction de la température et de la capacité d'ammoniac à produire. Le code est disponible en Annexes.%prog final matlab complet

