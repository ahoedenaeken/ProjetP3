\section{Conclusion}
Le but de cette première tâche était de créer un outil de gestion régissant une production d'ammoniac. 
Pour cela, nous nous sommes basés sur un flow-sheet simplifié, à partir duquel nous avons établi le bilan d'énergie du processus
ainsi qu'un bilan de matière. Pour le bilan d'énergie nous nous sommes uniquement focalisés sur le bloc "reformage primaire" ainsi
que sur le bloc "four" comme expliqué dans la section bilan d'énergie. Par la suite, nous avons fait un bilan de matière 
afin d'exprimer les quantités de matière necessaires pour une production voulue d'ammoniac. Une fois ces deux bilans 
réalisés, nous avons conçu un outil de gestion grâce au programme \textsc{Matlab}. L'outil de gestion permet de déterminer les 
quantités d'eau et de méthane nécessaires en fonction de deux variables, à savoir la capacité du plant et la température à 
la sortie du réacteur de reformage. En plus de cela, nous avons calculé le nombre de tuyaux nécessaires dans le réacteur
primaire de reformage pour une production journalière de $\unit{1500}{\tonne}$ à une température de sortie de 
$\unit{1080}{\kelvin}$. Les résultats obtenus nous semblent plausibles; ce qui nous permet de maintenant nous focaliser sur la 
seconde tâche.
