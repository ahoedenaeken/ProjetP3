\section{Conclusion}
Le but de cette première tâche était de créer un outil de gestion d'un la cadre d'une production d'ammoniac. 
Pour cela nous nous sommes basé sur un flow-sheet simplifié, à partir duquel nous avons fait un bilan d'énergie ainsi
qu'un bilan de matières. Pour le bilan d'energie nous nous sommes uniquement focalisé sur le bloc réformage primaire ainsi
que sur le bloc four comme expliqué dans la section bilan d'énergie. Par la suite nous avons analysé le bilan de matière 
afin d'exprimer les quantités de matières necessaires pour une production voulue d'ammoniac. Une fois ces deux bilan 
réaliser, nous avons conçus l'outil de gestion grâce au programme MatLab. L'outil de gestion permet de déterminer les débit 
de matières à chaque étapes du processus en fonction de deux variables, à savoir la capacité du plant et la temperature à 
la sortie du réacteur de réformage. En plus de cela, nous avons calculé le nombres de tuyaux nécessaires dans réacteur
primaire de réformage pour une production journalière de $\unit{1500}{\tonne}$ à une temperature de sortie de 
$\unit{1080}{\kelvin}$.
