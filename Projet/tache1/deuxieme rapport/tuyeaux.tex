\documentclass{article}
\usepackage[utf8]{inputenc}
\usepackage{natbib}
\usepackage[T1]{fontenc}
\usepackage[francais]{babel}
\usepackage{chemist}
\usepackage{array}
\usepackage[version=3]{mhchem}
\usepackage{amsmath}
\usepackage[squaren,Gray]{SIunits}
\usepackage{numprint}
\usepackage{amsfonts}
\usepackage{amssymb}
\usepackage{graphicx}
\usepackage{mathtools}
\usepackage{fullpage}
\usepackage{mhchem}
\usepackage{listings}
\usepackage{hyperref}
\usepackage{mathenv} %%%%%%%%%%% do not forget to add to head.tex
\usepackage{empheq} %%%%%%%%%%%% same
\author{Groupe 1246 }



\begin{document}
\subsection{Calcul du nombre de tuyaux dans le réacteur primaire de reformage}

Dans cette section, nous allons déterminer le nombre de tuyaux de diamètre de \unit{10}{\centi\meter}, que doit comporter le réacteur de reformage. On considère que la vitesse superficielle typique à l'entrée du réacteur est de \unit{2}{\meter\per\second}, dans le cadre d'une production journalière de \unit{1500}{t} d'ammoniac. On pose que le flux '$\dot{V}$' s'exprime de la sorte:

$\dot{V} = v \cdot A$, où $A$ est l'aire de la section du tuyaux valant $\unit{\pi \cdot 0.05^2}{\meter ^2}$. On trouve encore:

$\dot{V} = \cdot 2 \pi \cdot 0.05^2 = \unit{0.016}{\meter ^3 \per\second}$

Grâce à notre bilan de matière et en supposant que les gaz sont parfaits, nous pouvons trouver:

$F_{CH_4}$ = le flux de $CH_4$ en \unit{}{\meter ^3 \per\second}

$F_{H_2O}$ = le flux de $H_2O$ en \unit{}{\meter ^3 \per\second}

Les deux composés se déplaçant simultanément dans les tuyaux, le nombre de tuyaux $n$ se trouve grâce à la relation suivante:

$n = \dfrac{F_{CH_4} + F_{H_2O}}{\dot{V}}$
\end{document}
