\documentclass[11pt,a4paper]{report}
\usepackage[utf8]{inputenc}
\usepackage[francais]{babel}
\usepackage[T1]{fontenc}
\usepackage{amsmath}
\usepackage{amsfonts}
\usepackage{amssymb}
\usepackage{graphicx}
\usepackage{mathtools}
\usepackage{fullpage}
\author{Groupe 1246}
\usepackage[squaren,Gray]{SIunits}
\usepackage{numprint}
\usepackage{mhchem}
\usepackage{listings}

\begin{document}

\section*{Bilan d'énergie}

Cette section traite le bilan énergétique du procédé étudié.
Sur base du flow-sheet simplifié disponible ci-après, nous avons déterminé que seulement trois réactions
devaient être analysées pour réaliser un tel bilan: celle se produisant dans le bloc du four et celles dans 
le réacteur du reformage primaire.

\subsection*{Le bloc "Réacteur de reformage primaire"}

Dans ce bloc, nous devons traiter les réactions suivantes, prenant place simultanément, à une certaine température $T$:

\begin{itemize}
\item{\ce{CH_{4(g)} + H_{2}O_{(g)} <=> 3H_{2(g)} + CO_{(g)}}}
\item{\ce{CO_{(g)} + H_{2}O_{(g)} <=>  H_{2(g)} + CO_{2(g)}}}
\end{itemize}

\bigbreak
Nous ne nous intéresserons qu'à la première des deux; la démarche étant analogue pour la seconde. Nous savons que:
$$\Delta H\degree_{react1}(T)=3\Delta H\degree_{H_{2(g)}}(T) + \Delta H\degree_{CO_{(g)}}(T) - \lbrack\Delta H\degree_{CH_{4(g)}}(T) + 
\Delta H\degree_{H_{2}O_{(g)}}(T)\rbrack$$

Pour rappel, l'enthalpie standard d'une espèce chimique à une certaine température s'écrit: 
$$\Delta H\degree(T)=\Delta H\degree(T_{ref})  + \displaystyle \int_{T_{ref}}^{T} C_{p}(T) \mathrm{d}T$$. Les enthalpies à
une température de référence sont disponibles dans des tables. Il ne nous reste donc plus qu'à trouver l'expression
de $\Delta C_p$ en fonction de $T$.
Après quelques recherches\footnote{http://webbook.nist.gov/chemistry/form-ser.html}, nous avons trouvétrouvé la formule suivante:

$$\Delta C_p=A+\dfrac{BT}{1000}+C\left(\dfrac{T}{1000}\right)^2+D\left(\dfrac{T}{1000}\right)^3+
E\cdot\left(\dfrac{1000}{T}\right)^2$$

où $A,B,C,D,E$ sont des constantes propres à chaque composant (cfr. Annexe).
En remplaçant, nous pouvons facilement obtenir une expression pour $\Delta H^o_{react1}$, que nous ne retranscrirons pas
ici, mais qui est disponible dans le code \textsc{MATLAB} en Annexe. Pour une illustration avec quelques valeurs de 
température, se référer à la section "Analyse paramétrique".

\subsection*{Le bloc "Four"}
La réaction traitée dans cette partie est la suivante:

\ce{CH_{4(g)} + H_{2}O_{(g)} <=> 3H_{2(g)} + CO_{(g)}}

Nous savons que:

$\Delta H\degree_{react} = \Delta H\degree_{CO_{2(g)}} + 2\cdot \Delta H\degree_{H_{2}O_{(g)}} - \left(2\cdot \Delta H\degree_{H_{2(g)}} +
\Delta H\degree_{CH_{4(g)}}\right)$

Comme pour le réacteur primaire, le milieu réactionnel n'est pas à température ambiante, mais à la température $T$. 
Nous devons donc à nouveau trouver la valeur du $\Delta H\degree(T)$ de chaque composé grâce aux formules utilisées plus haut,
puis injecter ces valeurs dans l'équation du $\Delta H\degree_{react}$ -- ces calculs, vu leur longueur, ne sont pas repris dans
ce document.
Nous supposons que la réaction à la température $T$ sera exothermique car le four sert à fournir de l'énergie à la réaction
endothermique du réacteur primaire de reformage.
Nous posons: $\Delta H_{Reacteur} = a$ et $\Delta H_{Four} = b$, par mole de \ce{CH_4} dans les deux cas et $Q$, la quantité 
de \ce{CH_4} nécessaire dans le four.
On trouve alors:

$Q = \dfrac{a}{b}$



\section*{Annexe}
\subsection*{Données et constantes}
On connait les valeurs de $\Delta H\degree (\unit{25}{\celsius})$ pour les éléments suivants:

\begin{itemize}
\item{\ce{CH_{4(g)}} $\Rightarrow \unit{-74.6}{\kilo\joule\per\mole}$}
\item{\ce{H_2O_{(g)}} $\Rightarrow \unit{-241.83}{\kilo\joule\per\mole}$}
\item{\ce{CO_{(g)}} $\Rightarrow \unit{-110.53}{\kilo\joule\per\mole}$}
\item{\ce{CO_{2(g)}} $\Rightarrow \unit{-393.51}{\kilo\joule\per\mole}$}
\end{itemize}

On connaît les constantes permettant d'utiliser la formule $(2)$:

\begin{tabular}{|c|c|c|c|c|c|}
\hline 
\rule[-1ex]{0pt}{2.5ex}  & A & B & C & D & E \\ 
\hline 
\rule[-1ex]{0pt}{2.5ex} \ce{CH_{4(g)}} & -0.703 & 108.471 & -42.521 & 5.862 & 0.678 \\ 
\hline 
\rule[-1ex]{0pt}{2.5ex} \ce{H_2O_{(g)}} & 30.092 & 6.832 & 6.793 & -2.534 & 0.082 \\ 
\hline 
\rule[-1ex]{0pt}{2.5ex} \ce{CO_{(g)}} & 25.5675 & 6.0961 & 4.0546 & -2.6713 & 0.131 \\ 
\hline 
\rule[-1ex]{0pt}{2.5ex} \ce{CO_{2(g)}} & 34.2244 & 41.044 & -23.5297 & 5.5352 & -0.129 \\ 
\hline 
\rule[-1ex]{0pt}{2.5ex} \ce{H_{2(g)}} & 33.066 & -11.363 & 11.432 & -2.772 & -0.158 \\ 
\hline 
\end{tabular} 

\subsection*{Code \textsc{MATLAB}}
Voici une fonction en \textsc{MATLAB} créée par nos soins, nous permettant de calculer l'enthalpie d'un élément à une certaine
température $T$, en fonction des différentes constantes utilisées dans la formule $(3)$.
\lstset{language=Matlab,breaklines=true}
\begin{lstlisting}[frame=single]
function [Sol]= HmolT(Hfo,T2,A,B,C,D,E)
%% Hfo est le delta H formation a 298K en kJ/mol, T2 est la temperature pour laquelle on desire connaitre le nouveau delta H, A,B,C,D et E sont des donnees qui doivent etre prises sur le site NIST et qui dependent de la molecule pour laquelle on calcule le nouveau H.
%%
Intcp1 = A*298 + B* (298^2)/2000 + C*(298^3)/(3*1000^2) + D*(298^4)/(4*1000^3) + (-1000^2)*E/298;
= A*T2 + B* (T2^2)/2000 + C*(T2^3)/(3*1000^2) + D*(T2^4)/(4*1000^3) + (-1000^2)*E/T2;
Sol = Hfo*1000 + Intcp2 - Intcp1
end
\end{lstlisting}


\end{document}
