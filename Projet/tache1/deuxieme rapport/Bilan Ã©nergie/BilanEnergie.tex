\documentclass[11pt,a4paper]{report}
\usepackage[utf8]{inputenc}
\usepackage[francais]{babel}
\usepackage[T1]{fontenc}
\usepackage{amsmath}
\usepackage{amsfonts}
\usepackage{amssymb}
\usepackage{graphicx}
\author{Groupe 1246}
\usepackage[squaren,Gray]{SIunits}
\usepackage{numprint}
\usepackage{mhchem}
\begin{document}
\section*{Bilan d'énergie}
Il nous a été demandé de faire le bilan énergétique du procédé étudié. 
Sur base du flowsheet simplifié que nous avons complété en S2, nous avons déterminé que seulement trois réactions 
devaient être analysées pour réaliser le bilan: celles se produisant dans le bloc du four et dans le réacteur du 
reformage primaire.
\begin{itemize}
\item{$CH_{4(g)} + 2O_{2(g)} \Rightarrow 2H_{2}O_{(g)} + CO_{2(g)}$ (Combustion de $CH_{4(g)}$, réaction complète)}
\item{$CH_{4(g)} + H_{2}O_{(g)} \Rightarrow 3H_{2(g)} + CO_{(g)}$ (Reformage à vapeur de $CH_{4(g)}$, réaction à l'équilibre)}
\item{$CO_{(g)} + H_{2}O_{(g)} \Rightarrow H_{2(g)} + CO_{2(g)}$ (Reformage à vapeur de $CH_{4}$, réaction à l'équilibre)}
\end{itemize}

Nous avons commencé par calculer le $\Delta H$ de chaque équation.

\subsection*{Le bloc 'Réacteur de reformage primaire'}

Dans ce bloc, nous avons traité les réactions suivantes:
\begin{itemize}
\item{$CH_{4(g)} + H_{2}O_{(g)} \Rightarrow 3H_{2(g)} + CO_{(g)}$}
\item{$CO_{(g)} + H_{2}O_{(g)} \Rightarrow H_{2(g)} + CO_{2(g)}$}
\end{itemize}

Nous nous sommes d'abord interressés à la première des deux.
Nous savons que:

$\Delta H^o_{react1}=3\Delta H^o_{H_{2(g)}} + \Delta H^o_{CO_{(g)}} - (\Delta H^o_{CH_{4(g)}} + \Delta H^o_{H_{2}O_{(g)}})$

La réaction se passant dans un mileu à une température $T_2$ donnée, nous devons calculer les différences d'entalpie
molaire de formation des composés à cette température, sur base des tables de données expérimentales. Pour cela, nous 
avons utilisé la relation suivante:

$H^o_m(T_2)=\Delta H^o_{form}(T_1)+\int_{T_1}^{T_2} \Delta C_pdT$   (1)

En faisant des recherches, nous avons trouvé l'expression pour $\Delta C_p$ en fonction de $T$ suivante:

$\Delta C_p=A+\dfrac{BT}{1000}+C(\dfrac{T}{1000})^2+D(\dfrac{T}{1000})^3+\frac{E}{(\dfrac{T}{1000})^2}$        (2)

où $T$ est la température du mileu et A,B,C,D,E sont des constantes propres à chaque composant (cfr Annexe).

De $(1)$ et $(2)$, on trouve facilement:

$H^o_m(T_2)=\Delta H^o_{form}(T_1)+\int_{T_1}^{T_2} [A+\dfrac{BT}{1000}+C(\dfrac{T}{1000})^2+D(\dfrac{T}{1000})^3+\frac{E}{(\dfrac{T}{1000})^2}]$dT

ou encore

$H^o_m(T_2)=\Delta H^o_{form}(T_1) + [AT+\dfrac{BT^2}{2\cdot10^3}+C(\dfrac{T^3}{3\cdot10^6})+D(\dfrac{T^4}{4\cdot10^9})+\frac{-10^6E}{T}]^{T_2}_{T_1}$(3) 

En remplaçant la valeur des $\Delta H^o$ de chaque élément par leur valeur à la température $T$ obtenue avec $(3)$, on
peut facilement obtenir une expression pour $\Delta H^o_{react}$. À titre d'exemple, nous avons réalisé le calcul complet
pour un milieu réactionnel à une température standard de \unit{1080}{\kelvin}.

Comme indiqué plus haut, nous savons que:

$\Delta H^o_{react}=3\Delta H^o_{H_{2(g)}} + \Delta H^o_{CO_{(g)}} - (\Delta H^o_{CH_{4(g)}} + \Delta H^o_{H_{2}O_{(g)}})$

Calculons donc les enthalpies de formation des différents composés à la température de 1080 K, sur base de la formule 
$(3)$, des données précisées en annexe et des tables de données expérimentales à une température de \unit{298.5}{\kelvin}:

\begin{itemize}
\item{$\Delta H^o_{H_{2(g)}} (1080K)= \unit{23113}{\joule}$}
\item{$\Delta H^o_{CO_{(g)}} (1080K)= \unit{-86170}{\joule}$}
\item{$\Delta H^o_{CH_{4(g)}} (1080K)= \unit{-30556}{\joule}$}
\item{$\Delta H^o_{H_{2}O_{(g)}} (1080K)= \unit{-212480}{\joule}$}
\end{itemize}

En utilisant ces données nous trouvons:

$\Rightarrow \Delta H^o_{react1}= 3\cdot 23113 + (-86170) - (-30556 + (-212480))= \unit{226205}{\joule}$

de la même façon, nous trouvons pour la seconde réaction les résultats suivants:

\begin{itemize}
\item{$\Delta H^o_{CO_{2(g)}} (1080K)= \unit{-352750}{\joule}$}
\end{itemize}
$\Rightarrow \Delta H^o_{react2}= \Delta H^o_{CO_{2(g)}} + \Delta H^o_{H_{2(g)}} -(\Delta H^o_{H_{2}O_{(g)}} + \Delta H^o_{CO_{(g)}})$

=-352750 + 23113 - ( -212480 + (-86170)) = \unit{-30987}{\joule}7

Les deux réactions se produisant en même temps dans le même réacteur, nous pouvons établir un bilan énergétique 
général au bloc:


$\Delta H^o_{general}= \Delta H^o_{react1} + \Delta H^o_{react2} = 226205 - 30987 = \unit{195218}{\joule}$

\subsection*{Le bloc 'Four'}
La réaction traitée dans cette partie est la suivante:

$CH_{4(g)} + 2O_{2(g)} \Rightarrow 2H_{2}O_{(g)} + CO_{2(g)}$

Nous savons que:

$\Delta H^o_{react} = \Delta H^o_{CO_{2(g)}} + 2\cdot \Delta H^o_{H_{2}O_{(g)}} - (2\cdot \Delta H^o_{H_{2(g)}} + \Delta H^o_{CH_{4(g)}}) $

En supposant que la réaction se déroule à température ambiante, nous pouvons utiliser les table de $\Delta H^o$ pour une
température de \unit{298.5}{\kelvin}. Il vient:

$\Delta H^o_{react} = -393510 + 2\cdot -241830 - (-74600) = \unit{-802570}{\joule}$

La valeur trouvée nous indique que la réaction est exothermique. En effet, l'énergie dégagée par le four va être 
utilisée pour alimenter le réacteur de reformage primaire, dont le bilan énergétique est endothermique.
\section*{Annexe}
\subsection*{Données et constantes}
On connait les valeurs de $\Delta H^o$ pour les éléments suivants:
\begin{itemize}
\item{$CH_{4(g)} \Rightarrow \unit{-74.6}{\kilo\joule\per\mole}$}
\item{$H_2O_{(g)} \Rightarrow \unit{-241.83}{\kilo\joule\per\mole}$}
\item{$CO_{(g)} \Rightarrow \unit{-110.53}{\kilo\joule\per\mole}$}
\item{$CO_{2(g)} \Rightarrow \unit{-393.51}{\kilo\joule\per\mole}$}
\end{itemize}

On connaît les constantes permettant d'utiliser la formule $(2)$:

\begin{tabular}{|c|c|c|c|c|c|}
\hline 
\rule[-1ex]{0pt}{2.5ex}  & A & B & C & D & E \\ 
\hline 
\rule[-1ex]{0pt}{2.5ex} $CH_{4(g)}$ & -0.703 & 108.471 & -42.521 & 5.862 & 0.678 \\ 
\hline 
\rule[-1ex]{0pt}{2.5ex} $H_2O_{(g)}$ & 30.092 & 6.832 & 6.793 & -2.534 & 0.082 \\ 
\hline 
\rule[-1ex]{0pt}{2.5ex} $CO_{(g)}$ & 25.5675 & 6.0961 & 4.0546 & -2.6713 & 0.131 \\ 
\hline 
\rule[-1ex]{0pt}{2.5ex} $CO_{2(g)}$ & 34.2244 & 41.044 & -23.5297 & 5.5352 & -0.129 \\ 
\hline 
\rule[-1ex]{0pt}{2.5ex} $H_{2(g)}$ & 33.066 & -11.363 & 11.432 & -2.772 & -0.158 \\ 
\hline 
\end{tabular} 
\subsection*{Code matlab}
Voici une fonction en matlab créée par nos soins, nous permettant de calculer l'enthalpie d'un élément à une certaine
température, en fonction des différentes constantes utilisées dans la formule $(3)$.

function [Sol]= HmolT(Hfo,T2,A,B,C,D,E)

\% Hfo est le delta H formation a 298K en kJ/mol, T2 est la température pour

\% laquelle on desire connaitre le nouveau delta H, A,B,C,D et E sont

\% des donnees qui doivent etre prises sur le site nist et qui dependent

\% de la molecule pour laquelle on calcule le nouveau H.

$Intcp1 = A*298 + B* (298^2)/2000 + C*(298^3)/(3*1000^2) + D*(298^4)/(4*1000^3) + (-1000^2)*E/298;$

$ = A*T2 + B* (T2^2)/2000 + C*(T2^3)/(3*1000^2) + D*(T2^4)/(4*1000^3) + (-1000^2)*E/T2;$

$Sol = Hfo*1000 + Intcp2 - Intcp1$

end

\end{document}
