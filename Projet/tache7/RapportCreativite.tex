\section{Atelier d'initiation aux techniques de créativité en équipe}


\subsection{Le début de notre quête}

Le présent rapport résume de manière assez concise les points les plus importants de l'atelier de créativité du 5 novembre. Tout d'abord, nous ferons une contextualisation des enjeux actuels, qui déboucheront ensuite dans une démarche plutôt générale à suivre au moment d'essayer de trouver des solutions créatives à des problèmes actuels. Tout au long du rapport nous allons aussi donner quelques conseils importants, qui nous ont été donnés par les organisateurs. Enfin, nous conclurons avec les deux idées les plus importantes qui, mises en pratique, peuvent faciliter la tâche et faire de nous de véritables machines à innover.

Nous avons découvert que tout le monde est créatif(ve) à sa manière, ce qui fait que dans un groupe, la créativité est décuplée! Faisons une distinction entre les quatre caractères que l'on retrouve dans un bon groupe, en précisant d'ores et déjà qu'en réalité , chaque membre du groupe n’appartient pas à un unique type mais est une combinaison de chacun des types citées ci-dessous:

\begin{enumerate}
\item Le clarificateur: il est méthodique et ordonné
\item L'idéateur: il donne beaucoup d’idées mais il n'est pas structuré, tout comme ses idées.
\item  Le développeur: il recadre, se demande si les idées avancées sont réalisables.
\item Le réalisateur: il veut du concret, des réalisations, se concentre sur la phase convergente des idées.
\end{enumerate}

Nous pouvons chercher à définir ce qu’est la créativité mais en réalité, cette question est une mauvaise question! Si nous devions, malgré cela, donner une réponse, nous pourrions dire qu’être créatif consiste à produire des solutions originales et efficaces à un problème préalablement bien posé. 

Sans vouloir énumérer pour énumérer, nous pouvons faire remarquer qu'un processus créatif comprend deux phases: une phase \textit{divergente} où tout le monde donne ses idées, le groupe part "dans tous les sens", suivie d'une phase \textit{convergente} où il faut recadrer le tout et sélectionner les idées réalistes et appropriées.

\subsection{Où en sommes-nous au juste?}

Depuis quelques années, le métier d’ingénieur (et tous les métiers de façon générale) a connu un véritable bouleversement: l'arrivée d'un nouveau paradigme de développement. Ce changement représente une énorme différence par rapport à l'ancien, appelé dans ce rapport "le vieux paradigme". 

En effet, le vieux paradigme, valable depuis la belle époque du boum industriel (première et deuxième révolutions industrielles au XIX\textsuperscript{ième} siècle) voulait de l’ingénieur une amélioration de la production à tout prix, en réduisant les coûts, et sans se soucier vraiment des potentiels impacts négatifs --- tels que la contamination ou l'utilisation et la destruction excessive des ressources naturelles fournis par notre belle Terre Mère. Cela avait pour impact (en plus) que le progrès réalisé dans l'industrie était plutôt dû à une compétition acharnée et déchirante entre les travailleurs, se souciant peu de leur santé et de leur bien-être. Cependant, cette situation ne pouvait pas durer indéfiniment, l'homme prenant conscience de son rôle dans ce monde et des problèmes provoqués par cette méthode de production, il décida alors de changer.

C'est alors que le nouveau paradigme arriva: le développement durable. Les choses changèrent radicalement, et beaucoup des nouveaux paramètres à prendre en compte rentrèrent en jeu: la logique de compétition devint une logique de collaboration, les conséquences négatives devaient être réduites, ainsi qu'une utilisation minimale de ressources, le bien-être des travailleurs devint une des priorités (enfin, en tout cas, on en parle un peu plus), et un tas d'autres choses. Néanmoins, ceci apporta également une conséquence négative pour les ingénieurs: beaucoup de nouvelles "contraintes" provoquent parfois que les solutions apportées aux problèmes actuels ne soient pas toujours innovantes, séduisantes ou créatives.


\subsection{Alors, que faire pour apporter des solutions créatives?}

La réponse n'est pas aussi simple que l'on puisse le croire. Beaucoup diront que cela ne dépend que de l'ingénieur en question, et ils n'auront pas tout à fait tort. Cependant, une démarche "créative" peut sans aucun problème être utilisée, et alors, nous permettre à tous d'apporter des solutions uniques, utiles et qui répondent aux besoins actuels. Il faut avant tout réussir à se poser la \textit{bonne question}. C'est la partie la plus difficile et qui demande le plus d'imagination car la qualité de la solution apportée en dépend directement. Pour y arriver il faut mettre en place un \textit{commando créatif} \footnote{Groupe de personnes qui travailleront ensemble sur la recherche de la question et de la solution créative. En gros, c'est l’équipe d'ingénieurs et de spécialistes.}. Même si plusieurs rôles y sont impliqués, il est plus important de voir la façon de procéder:

\begin{enumerate}

\item \textit{Compréhension} du problème. Il faut tout d'abord faire une analyse objective du problème qui nous est posé et clarifier l'objectif: quelles sont les tâches que notre solution doit accomplir? (ex: produire ammoniac); quelles sont les différentes parties nécessaires au bon fonctionnement de notre projet? (ex: réacteurs, bassins de refroidissement, bureaux administratifs, postes de contrôle, parking, etc); quels services notre projet rend-il à la société? (ex: engrais, etc); quelles sont les ressources dont nous avons besoin? (ex: il nous faut une source de méthane, d'eau, d'air (et oui, on ne peut pas faire un site de production d'ammoniac sur la lune!) ou encore la taille du site de production) et enfin, quels sont les services externes dont nous avons besoin? (ex: une ligne de train, une autoroute, un hôpital, etc). Tous ces paramètres vont nous permettre d'avoir une idée de ce dont nous avons besoin et où nous pouvons implémenter notre solution.

\item \textit{Brainstorming} des idées. Si on ne donne une solution que sur le point vu précédemment, celle-ci risque d’être trop "conventionnelle" et classique (et on se fait virer). Il faut alors apporter un petit grain de folie, et c'est ici qu'il faut être créatifs. Au lieu de nous lancer aveuglement dans la première idée créative qui nous vient à l'esprit, il vaut mieux en choisir la meilleure parmi toute une série de potentielles idées. Pour ce faire il suffit de prendre notre problème de départ (ex: le site de production d'ammoniac) et choisir un (ou plusieurs) enjeu du développement durable \footnote{Par exemple, l'habitat, l’écologie, le bien-être, la collaboration, le dialogue inter-générationnel, ou plein d'autres!}, de préférence celui qui pourrait représenter le plus grand défit. C'est alors que la véritable réflexion commence: avec notre problème et l'enjeu, il faut réussir à faire une question: \textit{Comment faire pour (...)?} \footnote{\textit{La "question à la Disney"}} (ex: Comment faire pour transformer un site de production d'ammoniac en un pôle résidentiel esthétique, pratique et écologique?). Pour arriver à cette question il a fallu évidemment discuter en groupe autour de l'enjeu et apporter des nouvelles idées et voir vraiment ce qui pourrait être intéressant. Attention, il ne s'agit pas pour le moment de donner des solutions particulières mais seulement de discuter de la bonne question à se poser.

\item Recherche de \textit{pistes à exploiter}. Une fois la question bien posée, il faut donc déterminer toutes les potentielles pistes que nous pouvons exploiter pour donner des réponses. En effet, la solution au problème ne se comporte pas généralement d'une et une seule solution mais plutôt d'un ensemble. il s'agit donc ici de donner toute une série de potentielles réponses. Il faut évidemment ensuite discuter et choisir celles qui sont les plus créatives et plus innovantes.

\item Donner la \textit{solution}. C'est le but de notre travail: donner la réponse. Il s'agit alors de mettre ensemble les solutions "techniques" du problème et les idées innovantes révolutionnaires choisies. Ainsi, nous répondrons au mieux au cahier des charges qui nous a été donné et on ajoutera un petit extra qui répond aux enjeux de développement actuels.

\end{enumerate}

Voici donc les étapes plus générales, pas nécessairement obligatoires mais en tout cas utiles pour pouvoir donner une solution créative aux problèmes que l'on rencontrera au long de notre parcours. Il faut cependant faire deux commentaires. Tout d'abord, au moment de trouver des nouvelles idées, il s’agit évidemment d'en avoir un maximum, c'est la quantité qui est importante. Il ne faut donc pas hésiter à écrire tout ce qui nous passe par la tête, car qui sait, une des idées les plus farfelues peut s’avérer être la plus innovante. Deuxièmement, il est aussi intéressant d'avoir l'avis d'experts d'autres horizons, de trouver l'inspiration ailleurs. D'une façon ou une autre, nous sommes "formatés" en tant qu’ingénieurs à donner la solution la plus pratique mais nous n'avons pas nécessairement l'habitude de donner la plus innovante, il faut une source d'inspiration externe qui nous permettra d'en avoir une!


\subsection{Trop de blabla, il faut retenir quoi?}

Même si la démarche à suivre peut être la clé pour résoudre différents problèmes, il faut retenir deux choses de cet atelier. 

Tout d'abord, se poser la bonne question est déjà donner une bonne réponse. D’où l’intérêt d'en trouver une qui soit suffisamment audacieuse pour être innovante mais également réaliste (n'oublions pas, le but est de concevoir quelque chose de concret, pas de divaguer dans le vide). Ensuite, le deuxième point important c'est de chercher des sources d'inspiration ailleurs, ne pas rester enfermés dans notre zone de confort. Il ne faut pas avoir peur d'explorer de nouvelles solutions, et pour ce faire, il est impératif de ne pas voir les enjeux du développement durable comme une contrainte mais plutôt comme une nouvelle voie d'exploration, qui peut nous être plus utile que ce que nous le croyions: c'est véritablement une nouvelle porte qui s'ouvre devant nous!
