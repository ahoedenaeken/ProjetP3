\documentclass[12pt]{amsart}
\usepackage[latin1]{inputenc}  
\usepackage[T1]{fontenc}
\usepackage{geometry} 
\geometry{a4paper} 


\title{Visite du centre de recherche Total à Feluy} 
\author{Colpin Lionel}
\date{12/11/2014} 

\begin{document}

\maketitle

\section{Accueil et réponse aux questions}
Lors de notre arrivée au centre de recherche, nous avons été accueilli par une présentation générale de l'entreprise. Celle-ci nous informait de manière globale sur le groupe Total et leurs activités. 

Nous avons ensuite eu des réponses à une partie des questions, plus spécialement sur les catalyseurs utilisés. Nous avons appris que sans les catalyseurs, la plupart des réactions effectuée par le groupe Total seraient impossible. On ne parle même pas d'une simple vitesse de réaction qui serait trop lente pour être rentable mais bien d'une impossibilité de faire la réaction. Au cours de la réaction de polymérisation, le catalyseur sers de support aux autres molécules qui formeront le polymère. Comme au fil des ans les catalyseurs ont été améliorés, ces derniers sont laissés dans la structure produite. On a donc une consommation du catalyseur utilisé. Il apparait donc essentiel de continuer à améliorer les catalyseurs dans le but d'en diminuer la quantité nécessaire à la synthèse des polymères.

La dernière partie de la présentation fut tournée sur l'écologie en nous expliquant les différents efforts de la société pour s'orienter de plus en plus vers une production d'énergie plus verte. Une des pistes actuellement en cours est l'utilisation de bio-carburants produits à partir des végétaux. Le problème majeur est que la production de ces bio-carburant se fait à partir des ressources utilisées pour l'alimentation. Un schéma idéal serait de pouvoir produire les bio-carburant à partir des déchets issus de l'alimentaire.

\section{Visite des batiments}

\subsection{Visite de l'unité pilote de production des produits finis}

Nous avons commencé la visite des locaux par l'unité pilote de production des produits finis. Il s'agit de la dernière étape de production durant laquelle on utilise les polymères dans le but de synthétiser la demande du client : gobelets, sacs plastiques, etc. Le procédé utilisé est toujours le même pour le début : on fait fondre la poudre obtenue lors de la synthèse du polymère pour injecter la pâte ainsi obtenue dans le moule désiré. On laisse ensuite refroidir puis on éjecte le produit du moule.

\subsection{Visite du centre de contrôle de l'unité pilote}

Nous avons poursuis la visite par le centre de contrôle de l'unité pilote. L'unité pilote consiste en une sorte d'usine miniature. Cette dernière permet de faire des essais sur les procédés de production (variation des paramètres comme la pression, le dosage de catalyseur, etc) à une échelle plus petite que celle de l'usine. Cela a l'avantage d'être plus souple en cas de problème. En effet, comme on observe les problèmes à une plus petite échelle, on peut les régler plus facilement en évitant de stopper la production de l'usine principale. Il faut bien se dire que si on a un problème sur l'unité pilote, c'est gênant alors qu'un problème à l'usine, c'est vraiment problématique car si il faut bouger un bloc de 50 kilos qui bouche le réacteur dans l'unité pilote, il s'agit de déloger un  bloc de quelques tonnes dans le réacteur de l'usine entrainant un arrêt de la production.

\subsection{Visite du laboratoire}

L'étape suivante de notre visite fut le passage par le laboratoire dans lequel sont élaborés et testé des nouveaux procédés. Le laboratoire offre une souplesse encore plus grande que l'unité pilote en matière de test des nouveaux procédés. Le laboratoire peut procéder à de très nombreux tests en simultané dans le but de tester, par exemple, des nouvelles formules de catalyseurs. 

\subsection{Visite des modules pilotes}

Nous avons cloturé la visite par le passage par la salle où se situent les modules pilotes. Ces modules pilotes sont des installations utilisées par le laboratoire pour tester les procédés. Ces modules sont complètement automatisés et peuvent tourner en 24h/24 et en 7j/7. La salle dans laquelle sont situés les modules est en dépression par rapport au reste du laboratoire pour permettre de cantonner les éventuelles fuites de gaz à cette salle. La salle est également équipée de multiples capteurs qui assurent en permanence la sécurité de la salle.


\end{document}
