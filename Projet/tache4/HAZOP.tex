\documentclass[10pt,a4paper]{article}
\usepackage[utf8]{inputenc}
\usepackage{natbib}
\usepackage[T1]{fontenc}
\usepackage[francais]{babel}
\usepackage{chemist}
\usepackage{array}
\usepackage[version=3]{mhchem}
\usepackage{amsmath}
\usepackage[squaren,Gray]{SIunits}
\usepackage{numprint}
\usepackage{amsfonts}
\usepackage{amssymb}
\usepackage{graphicx}
\usepackage{mathtools}
\usepackage{fullpage}
\usepackage{mhchem}
\usepackage{listings}
\usepackage{hyperref}
\usepackage{tabularx}
\usepackage{lscape}



\begin{document}

\section{Questions}

\begin{enumerate}

    \item Quels sont les dangers présentés par les substances mises en œuvre durant la synthèse de l’ammoniac?
    
    \item Comment les flux de matière circulent-ils dans la section en question et au sein du réacteur de conversion?
          Utilisez des surligneurs pour indiquer ces circulations sur le PFD et sur les PIDs de la section. Seules
          les circulations concernant le procédé en régime seront prises en compte.
    
    \item Pourquoi n’y a-t-il pas de soupape de sécurité ou de disque de rupture (les deux types de dispositifs
          servent à protéger un équipement ou une ligne contre les surpressions) sur le réacteur de synthèse du \ce{NH_3}?
    
    \item Pourquoi y a-t-il des disques de rupture sur l’échangeur 124-MC?
    
\end{enumerate}

\section{Réponses}

\subsection{Réponse 1}

La tableau ci-dessous reprend les trois substances mises en œuvre durant la synthèse de l’ammoniac et
donne une idée des dangers.



\begin{tabular}{|c||p{4cm}|p{4cm}|p{4cm}|}
\hline 
 & Diazote \ce{N_2} & Dihydrogène \ce{H_2} & Ammoniac \ce{NH_3} \\ 
\hline 
\hline
Stabilité & Très stable & Composant extrêment inflammable & Inflammable \\ 
\hline 
Nocivité & Risque d'anoxie & Pas toxique & Toxique par inhalation, brûlure au contact direct \\ 
\hline 
Environnement & Aucune dégradation de l'environnement & Risque majeur pour la couche d'ozone & Toxique, aucun effet
au niveau du réchauffement climatique \\ 
\hline 
Stockage & Le réservoir doit pouvoir supporter une grosse détente & Les normes imposent un stockage externe & Conservation
à l'écart de toute flamme et bonne ventilation du lieu de stockage imposée (fuite) \\ 
\hline 
Détection & Capteur électronique & Capteur électronique & Odeur particulière, baguette souffrée \\ 
\hline 
\end{tabular} 

\subsection{Réponse 2}

Voir séance.

\subsection{Réponse 3}

Nous n'avons pas de risque de surpression. Premièrement, étant donné que la réaction de formation de \ce{NH_3} tend à
réduire le nombre de môles de gaz, elle tend à réduire la pression. Par conséquent, nous n'aurons pas d'augmentation de
pression en activité, il n'est donc pas nécessaire de gérer cette augmentation. Deuxièmement, nous pouvons mentionner le
fait qu'une augmentation de la pression va seulement provoquer une augmentation de la vitesse de déplacement du gaz, étant
donné qu'il circule "librement" sur toute cette partie du circuit.

\subsection{Réponse 4}

Si un des tuyaux se brise, à cause de sa pression interne par exemple, son contenu va se répandre et donc la pression
maximale que le shell (la coquille, la coque) peut contenir, sera dépassée. Ceci provoquera une explosion du shell, ce que
les disques de rupture permettent justement d'éviter. Les disques de rupture fonctionnent de la manière suivante: ils
redirigent le contenu du shell vers un tank sécurisé, "a safe place".

L'échangeur 124MC peut être assimilé à un radiateur: il est compose d'un shell et d'une série de tubes qui le
traversent, le but étant un échange de chaleur entre le contenu du shell et les tubes. Le shell est rempli d'eau
(a priori elle reste liquide) et elle peut supporter une pression de plus au moins \unit{15}{bar}. Les tubes par contre
contiennent le gaz qui vient du réacteur, ce gaz se trouve à \unit{150}{bar} environ. Dès lors, si un de ces tuayx venait
à se briser, à cause de sa pression interne par exemple, son contenu va se répandre et donc la pression maximale que le
shell (la coquille, la coque) peut contenir, sera dépassée. Ceci provoquera une explosion du shell, ce que les disques de
rupture permettent justement d'éviter. Les disques de rupture fonctionnent de la manière suivante: lorsque le tube se casse
et la pression du shell va augmenter, les disques redirigent le contenu du shell vers un tank sécurisé, "a safe place". En
d'autres mots, ce sont les disques qui exploseront avant que le shell ne le fasse, et ils permettront alors de rediriger le
contenu du shell (qui est alors un mélange gaz/eau) vers un endroit sûr.

\begin{landscape}
\begin{table}
\centering

\begin{tabular}{|p{2cm}|p{3cm}|p{4cm}|p{6cm}|p{5cm}|}
\hline 
Noeud & Mot guide & Cause & Conséquence & Mesures \\ 
\hline 
\hline

 Entrée A2 Réacteur 105MD & LESS/NO FLOW & Vanne HV1046 pas assez ouverte & Mauvais refroidissement au sein du
 réacteur & Alarme \\ 

 & & & & \\

 & & & Augmentation de la température dans le réacteur & Dispositif d'ouverture automatique \\ 
 
 & & & & \\

 & & & Augmentation dans la sortie B du réacteur &  \\ 
 
 & & & & \\
 
 & MORE FLOW & Vanne HV1046 trop ouverte & Plus de refroidissement au niveau des lits & Alarme \\
 
 & & & Diminution de la température au sein du réacteur & Dispositif de fermeture automatique \\
 
 & & & & \\
 
 & & & Pas de conséquence en terme de sécurité & \\
 
 & & & & \\
  
 & REVERSE FLOW & pas considéré & pas considéré & pas considéré \\
  
  
 & & & & \\
 & & & & \\
 & & & & \\

 Shell échangeur calorifique 124MC & MORE PRESSURE & Rupture du tube interne (le gaz se répend dans l'enceinte de
 l'échangeur & Contamination & Alarme \\
 
  & & & & \\

  & & & Risque d'explosion de l'échangeur  & Activation du disque de rupture + Déversement dans réservoir alternatif sûr \\
 
  & & & & \\

  & LESS PRESSURE & Fuite dans le shell & Perte d'eau & Alarme \\
 
  & & & & \\

  & & & Mauvais refroidissement & Réparation \\
  
   & & & & \\

 
 \hline


\end{tabular} 
\caption{HAZOP}
\end{table}
\end{landscape}

\end{document}
