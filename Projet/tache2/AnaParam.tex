
\section*{Comparaison entre le modèle de notre outil de gestion et celui de Aspen+}
Il nous a été demandé dans le cadre de la seconde tâche d'analyser plus précisément la dernière étape du procédé, le réacteur de production d'ammoniac, considéré comme une "une boite noire" depuis le début du projet. Une analyse paramétrique a été réalisée avec notre outil de gestion ainsi qu'avec le logiciel Aspen+. Ce document présente les résultats et conclusions de cette dernière ainsi que la comparaison des modèles obtenus à l'aide des deux procédés.


\subsection*{Comparaison entre les deux modèles}
Dans les deux cas, nous remarquons que l'influence de la tempèrature et de la pression font évoluer le rendement dans un même sens: une augmentation de pression et une diminution de la température optimise ce dernier.
Dans les deux cas, une purge est obligatoire vu l'argon, dont la présence déplace l'équilibre au sein du réacteur. Sans quoi, la réaction ne saurait plus possible à partir d'un certain nombre de réinjection. Les résultats obtenus dans le cas des deux modèles coïncident, comme le montre les graphiques suivants:

\begin{figure}[ht!]
 \centering
 \includegraphics[scale=0.6]{graphe1.png}
 \caption{Rendement en fonction de la température}
 \label{scheme}
\end{figure}

\begin{figure}[ht!]
 \centering
 \includegraphics[scale=0.6]{graphe2.png}
 \caption{Rendement en fonction de la pression}
 \label{scheme}
\end{figure}

Les différences entre le résultat obtenu avec Aspen+ et notre outil de gestion peuvent provenir de plusieurs éléments comme les imprécision calculatoire de l'un ou l'autre logiciel. Du plus il peut aussi être dû au fait que la simulation réalisée avec Aspen+ prenait plus éléments en compte que notre outil de gestion, comme la pression présente dans les différents tuyaux. Ces restent pas ailleurs relativement petite dans la mesure où ce sont les même relation réactionnelles qui ont été utilisée.


