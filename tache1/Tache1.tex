\documentclass[11pt,a4paper]{report}
\usepackage[utf8]{inputenc}
\usepackage[francais]{babel}
\usepackage[T1]{fontenc}
\usepackage{amsmath}
\usepackage{amsfonts}
\usepackage{amssymb}
\usepackage{graphicx}
\author{Groupe 1246}
\title{Projet P3 LFSAB1503: Rapport de la première tâche}
\begin{document}
\maketitle
\section{Équation de la réaction et bilan de matière}
Il nous est demandé de rechercher la quantité nécéssaire des différents composés nécessaire à la synthèse de l'ammoniac. Il nous était dit que l'ammoniac pouvait être obtenu à partir de dihydrogène ($H_2$) et de diazote ($N_2$). Npus sommes donc débouché sur l'équation de synthèse de l'ammoniac suivante: \\
$N_2 + 3H_2 \Rightarrow 2NH_3$ \\
La masse molaire de l'ammoniac étant de 17g/m, nous en avons déduit que 1000 tonnes correspondaient à $\dfrac{10^{9}}{17}$ moles. Nous avons ensuite fait un tableau d'avancement de la réaction: \\
\begin{tabular}{|c|c|c|c|}
\hline 
 & $N_2$ & $3H_2 \Rightarrow$ & $2NH_3$ \\ 
\hline 
Initial & $\dfrac{10^{9}}{17}*\dfrac{1}{2}$ & $\dfrac{10^{9}}{17}*\dfrac{3}{2}$ & 0 \\ 
\hline 
Réaction & -$\dfrac{10^{9}}{17}*\dfrac{1}{2}$ & -$\dfrac{10^{9}}{17}*\dfrac{3}{2}$ & +$\dfrac{10^{9}}{17}$ \\ 
\hline 
Final & 0 & 0 & $\dfrac{10^{9}}{17}$ \\ 
\hline 
\end{tabular} 
\\
Où les données sont données en moles.
\\
La réaction se produisant en continu, on peut calculer des flux de quantité pour une période de 24 heures:\\
On obtient selon nos calculs:\\ une consommation de $N_2$ égale à: \\
$\Rightarrow \dfrac{\dfrac{10^{9}}{17}*\dfrac{1}{2}}{3600*24} \cong 340.41 $moles/s.\\
une consommation de $H_2$ égale à: \\
$\Rightarrow \dfrac{\dfrac{10^{9}}{17}*\dfrac{3}{2}}{3600*24} \cong 1021.241 $moles/s.\\
une production de $NH_3$ égale à: \\
$\Rightarrow \dfrac{\dfrac{10^{9}}{17}}{3600*24} \cong 680.827 $moles/s.\\

\section{Aspect thermique}
Selon nos recherche, nous avons trouvé que la réaction était exothermique ($\Delta H_{react} = -92.2kJ$/mole). Il nous était indiqué que la température du réacteur devait être maintenue à 500 degrés Celsius et que celui-ci, vu le caractère exothermique de la récation, pouvait être refroidi par un débit continu d'eau, dont la température variait entre 25 et 90 degrés Celsius.

\subsection{Calcul de volume d'eau nécessaire (pour une mole produite}
Nous savons donc que: $\Delta H_{react} = -92.2kJ$/mole. \\ \\
Nous savons aussi que:\\
 $q = m*C*dT$ \\
 où\\
 $\Rightarrow$ C est la constante calorifique massique de l'eau valant 4.18 ($\dfrac{J}{Celsius *g}$). \\
$\Rightarrow$  m est la masse total du volume d'eau.\\
\\
 Vu les indication données et, on peut facilement trouver que $dT=65$. En supposant que la température initiale de réacteur est de 500 degrés Celsius, il vient: \\
$92200 = 4.18*65*m \Rightarrow \dfrac{92200}{4.18*65} = m = 339.344$ g, qui correspond à 0.339344l d'eau.

\subsection{Calcul du débit d'eau nécessaire}
Nous avions calculé plus haut que le rythme de production de $NH_3$ était de environ 680.827 moles/s, il vient donc: \\
$680.827*0.339344 = 231.03$ \\
le débit d'eau nécessaire serait donc de 231.03 l/s.

teset
\end{document}
